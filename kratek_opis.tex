\documentclass[a4paper,12pt]{article}
\usepackage[slovene]{babel}
\usepackage[utf8]{inputenc}
\usepackage[T1]{fontenc}
\usepackage{lmodern}
\usepackage{amsmath}
\usepackage{amssymb}
\usepackage[shortlabels]{enumitem}
\usepackage{graphicx}

\newtheorem{definition}{Definicija}

\pagestyle{plain}

\begin{document}
\author{Miha Jan in Sara Žužek}
\date{December 2023}
\title{Weak k-Metric Dimension \\ (kratek opis)}
\maketitle

\section{Navodilo naloge}
Implement an ILP model for this invariant, and then write separate
small programs in Sage to answer each of following questions by exhaustive search.
\begin{enumerate}
    \item Find graphs for which $wdim_k(G) = \Delta(x, y)$ for a pair of vertices $x, y \in V(G)$ such that
    $d(x, y) \geq 3$.
    \item Determine $\kappa(G)$ and $wdim_k(G)$ for Cartesian products of cycles $G = C_a \square C_b$.
    \item  Determine the graphs G with $wdim_k(G) = dim_k(G)$ for various k with $k \leq \kappa(G)$.
\end{enumerate}
For small graphs, apply a systematic search; for larger ones, apply some stochastic search.


\section{Uporabne definicije}
    \begin{definition}
       Naj bo $S \subseteq V(G)$ in $a, b \in V(G) \cup E(G)$. Definiramo $\Delta_S (a,b)$ kot vsoto razlik razdalj od $a$ in $b$ do vsakega vozlišča $S$. 
       Torej je $$\Delta_S (a,b) = \sum_{s \in S } |d(s,a) - d(s,b)|.$$
       Označimo $\Delta_{V(G)} (a,b) = \Delta (a,b)$.
    \end{definition}

    

    \begin{definition} 
        {\bf Šibka (vozliščna) k-metrična dimenzija} grafa $G$ $wdim_k(G)$, je kardinalnost/moč
        najmanjše množice vozlišč $S$ grafa $G$, tako da za vsak par vozlišč $x,y \in V(G)$ velja $\Delta_S (x,y) \geq k$.
    \end{definition}

    \begin{definition}
        Največja vrednost parametra $k,$ za katerega je Šibka  k-metrična dimenzija grafa G smiselno definirana označimo z $\kappa(G)$ 
    \end{definition}

    \begin{definition}
        {\bf K-metrična dimenzija} grafa $G$ $dim_k(G)$ je velikost najmanjše množice vozlišč $S$ grafa $G$, ki reši graf $G$ in ji rečemo k-rešljiva množica. 
        Za razliko od standardne metrične dimenzije ta zahteva, da vsak par vozlišč reši vsaj k vozlišč. K-metrična dimenzija se ujema z običajno dimenzijo, ko je $k = 1$.
    \end{definition}
\section{Opis problema} 
Najina celotna projektna naloga se bo navezovala na k-te šibke dimenzije grafov. Kot glavno gradivo nama bo služil čanek \cite{peterin2023resolving}.

Projekt bova razdelila na več manjših delov potem pa bova za vsakega od njih napisala celoštevilski linearni program (CLP), ki bo rešil posamezne dele problema. 

\begin{enumerate}
    \item V tem delu bova iskala grafe za ketere velja $wdim_k(G) = \Delta (x,y)$, pri čemer je $d(x,y) \geq 3$ za izbrani vozlišči $x, y \in V(G)$.

    \item Določila bova $\kappa(G)$ in $wdim_k(G)$ za kartezične produkte ciklov $G = C_a \square C_b$.
    
    \item Za različne vrednosti $k$ morava najti grafe G za katere velja lastnost $wdim_k(G) = dim_k(G)$, pri čemer $k \leq K(G)$.
\end{enumerate}

\section{Načrt dela}
Za pisanje CLP bova uporabljala okolje Sage (SageMath), ki ima vgrajeno podporo za pisanje CLP. V prvem delu se bova osredotočila predvsem na pisanje učinkovitega CLP, ki bo deloval na manjših grafih. Ugotoviti morava kako smisleno izbrati spremenljivke, ki bodo v njem nastopale in jih potem smiselno minimizirati. 

V nadaljevanju bova poskušala uporabiti rezultate iz prvega dela in to implementirati na večjih grafih s pomočjo metahevristike.




\begin{thebibliography}{99}
    \bibitem{peterin2023resolving}
    I. Peterin, J. Sedlar, R. Škrekovski, I. G. Yero,
    \emph{Resolving vertices of graphs with differences},
    (2023) arXiv preprint arXiv:2309.00922.
    \end{thebibliography}

\end{document}