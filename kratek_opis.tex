\documentclass[a4paper,12pt]{article}
\usepackage[slovene]{babel}
\usepackage[utf8]{inputenc}
\usepackage[T1]{fontenc}
\usepackage{lmodern}
\usepackage{amsmath}
\usepackage{amssymb}
\usepackage[shortlabels]{enumitem}
\usepackage{graphicx}

\newtheorem{definition}{Definicija}

\pagestyle{plain}

\begin{document}
\author{Miha Jan in Sara Žužek}
\date{December 2023}
\title{Weak k-Metric Dimension \\ (kratek opis)}
\maketitle

\section{Navodilo naloge}
Implement an ILP model for this invariant, and then write separate
small programs in Sage to answer each of following questions by exhaustive search.
\begin{enumerate}
    \item Find graphs for which $wdim_k(G) = \Delta(x, y)$ for a pair of vertices $x, y \in V(G)$ such that
    $d(x, y) \geq 3$.
    \item Determine $\kappa(G)$ and $wdim_k(G)$ for Cartesian products of cycles $G = C_a \square C_b$.
    \item  Determine the graphs G with $wdim_k(G) = dim_k(G)$ for various k with $k \leq \kappa(G)$.
\end{enumerate}
For small graphs, apply a systematic search; for larger ones, apply some stochastic search.


\section{Uporabne definicije}
    \begin{definition}
       Naj bo $S \subseteq V(G)$ in $a, b \in V(G) \cup E(G)$. Definiramo $\Delta_S (a,b)$ kot vsoto razlik razdalj od $a$ in $b$ do vsakega vozlišča $S$. 
       Torej je $$\Delta_S (a,b) = \sum_{s \in S } |d(s,a) - d(s,b)|.$$
       Označimo $\Delta_{V(G)} (a,b) = \Delta (a,b)$.
    \end{definition}

    

    \begin{definition} 
        {\bf Šibka (vozliščna) k-metrična dimenzija} grafa $G$ $wdim_k(G)$, je kardinalnost/moč
        najmanjše množice vozlišč $S$ grafa $G$, tako da za vsak par vozlišč $x,y \in V(G)$ velja $\Delta_S (x,y) \geq k$.
    \end{definition}

    \begin{definition}
        Največja vrednost parametra $k,$ za katerega je Šibka  k-metrična dimenzija grafa G smiselno definirana označimo z $\kappa(G)$ 
    \end{definition}

    \begin{definition}
        {\bf K-metrična dimenzija} grafa $G$ $dim_k(G)$ je velikost najmanjše množice vozlišč $S$ grafa $G$, ki reši graf $G$ in ji rečemo k-rešljiva množica. 
        Za razliko od standardne metrične dimenzije ta zahteva, da vsak par vozlišč reši vsaj k vozlišč. K-metrična dimenzija se ujema z običajno dimenzijo, ko je $k = 1$.
    \end{definition}
\section{Opis problema} 
Najina celotna projektna naloga se bo navezovala na k-te šibke dimenzije grafov. Kot glavno gradivo nama bo služil čanek \cite{peterin2023resolving}.

Projekt bova razdelila na več manjših delov. Napisala bova CLP, ki bo za dan graf določil množico $S,$ katere moč bo predstavljala šibko dimenzijo obravnavanega grafa. Potem bova ločila najino delo na 3 primere glede navodila

\begin{enumerate}
    \item V tem delu bova iskala grafe za ketere velja $wdim_k(G) = \Delta (x,y)$, pri čemer je $d(x,y) \geq 3$ za izbrani vozlišči $x, y \in V(G)$.

    \item Določila bova $\kappa(G)$ in $wdim_k(G)$ za kartezične produkte ciklov $G = C_a \square C_b$.
    
    \item Za različne vrednosti $k$ morava najti grafe G za katere velja lastnost $wdim_k(G) = dim_k(G)$, pri čemer $k \leq \kappa(G)$. V tem delu bova uporabila algoritem, ki ga je napisala skupina 4 in izračuna $dim_k(G).$ Na istem grafu pa bova pognala še najin CLP in opazovala za kakšne grafe se ti dimenziji ujemata.
\end{enumerate}

\section{Načrt dela}
Za pisanje CLP bova uporabljala okolje Sage (SageMath), ki ima vgrajeno podporo za pisanje CLP. V prvem delu se bova osredotočila predvsem na pisanje učinkovitega CLP, ki bo deloval na manjših grafih. Ugotoviti morava kako smisleno izbrati spremenljivke, ki bodo v njem nastopale in jih potem smiselno minimizirati. 

V nadaljevanju bova poskušala uporabiti rezultate iz prvega dela in to implementirati na večjih grafih s pomočjo metahevristike.

\subsection{Načrt dela za majhne grafe (sistematično)}
Prva naloga je, da napiševa CLP. Tega se bova lotila tako, da bova v Sage prepisala sledeči CLP (ideja zanj je zapisana v \cite{peterin2023resolving}):

Naj bo $V(G)=\left\{v_{1}, \ldots, v_{n}\right\}$ množica vozlišč grafa $G$. Naj bo $S \subseteq V(G)$. Definiramo celoštevilske spremenljivke $x_{i}$ za vsako vozlišče grafa po sledečem predpisu 

$$
x_{i}= \begin{cases}1 & \text { if } v_{i} \in S \\ 0 & \text { if } v_{i} \notin S\end{cases}
$$

iskani CLP je naslednje oblike

$$
\begin{array}{ll} 
& \min \sum_{i=1}^{n} x_{i} \\
\text { s.t. } & \sum_{i=1}^{n}\left|d\left(v_{a}, v_{i}\right)-d\left(v_{b}, v_{i}\right)\right| \cdot x_{i} \geq k \text { za vsak par } v_{a}, v_{b} \in V(G), \\
& x_{i} \in\{0,1\} \text { za vse } 1 \leq i \leq n .
\end{array}
$$

Ob rešitvi zgornjega CLP nam bodo spremenljivke $x_{i}$ povedale katera vozlišča so v množici $S$ (če je $x_{i}=1$, potem $v_{i} \in S$). Torej potem preprosto razberemo tudi moč $S$, kar je iskana $wdim_k(G)$.

Ob proučevanju zgornjega CLP so se pojavila naslednja vprašanja (in opažanja):
\begin{itemize}
    \item Kako definirati dodatne spremenljivke, da se bova lahko znebila absolutnih vrednosti v CLP.
    \item v članku je navedena ocena $k \leq wdim_k(G).$ Ali bi torej z dodatnim pogojem  $\sum_{i=1}^{n} x_{i} \geq k$ pospešili iskanje rešitve? (Pogoj bi omejil moči množic $S$)
    \item V CLP bo potrebno izračunati razdalje med vsemi pari vozlišč. To bi naredila z vgrajeno funckijo v sage z Floyd-Warshallovim algoritmom (št. operacij $O(n^3)$)
\end{itemize}

\subsubsection{Posebnosti za 1. del naloge}
Za vsak grah $G,$ ki ga bova obravnavala bo seveda treba izračunati $wdim_k(G)$ za vse smiselne $k$.

Poleg tega bo seveda treba izračunati $\Delta(x,y) = \sum_{s \in V(G) } |d(s,x) - d(s,y)|$ za vse pare vozlišč. Tega bi se lotila na sledeči način:
\begin{itemize}
    \item Najprej izvedeva FW algoritem na grafu, da dobiva tabelo razdalj med vsakim parom vozlišč. (Matrika bo simetrična, ker so grafi neusmerjeni-to bo morda koristilo.)
    \item Z dvojno zanko se sprehodimo čez vse pare vrstic matrike dobljene prek FW in sproti pišemo novo matriko, recimo da jo označimo $A$ (tudi A bo simetrična). Na (i,j)-tem mestu A bo izračun $\Delta(i,j)$, ta je izračunan prek vsote razlik absolutnih vrednostih ob sprehodu čez stolpce vrstic i in j.
    \item prek $A$ bomo v 2. delu naloge dobili tudi $\kappa(G),$ ki je kar minimalni element matrike $A$
    \item zaradi pogoja $d(x,y) \geq 3$ bova obravnavala (opazovala ali obstaja enakost z šibko dimenzijo) vrednosti v matriki $A$ le za tiste elemente za katere je istoležna vrednost v matriki FW  $\geq 3$ 
    \item z nekaj if in for zankami se bova torej sprehodila čez tabelo in opazovala ali graf zadošča tem pogojem sicer nadaljevala na naslednjih 
\end{itemize}

\subsubsection{Posebnosti za 2. del naloge}
Enako kot je opisano v prejšnjem podrazdelku bova izračunala $\kappa(G).$ Zaradi simetrij kartezičnih produktov ciklov bi bilo smiselno ugotoviti kako bi lahko  le te uporabila in zmanjšala časovno zahtevnost programa.

\subsubsection{Posebnosti za 2. del naloge}
Najprej bi želela določiti $\kappa(G),$ da bova vedela za katere k je smiselno dimenzije sploh računati. Potem bi na istem grafu računala svojo šibko dimenzijo z najinim CLP in navadno k-to dimenzijo z uporabo kode skupine 4. 

\begin{thebibliography}{99}
    \bibitem{peterin2023resolving}
    I. Peterin, J. Sedlar, R. Škrekovski, I. G. Yero,
    \emph{Resolving vertices of graphs with differences},
    (2023) arXiv preprint arXiv:2309.00922.
    \end{thebibliography}

\end{document}